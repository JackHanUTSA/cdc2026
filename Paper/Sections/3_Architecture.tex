\section{Architecture}\label{architecture}

 

We propose a novel architecture called Active Learning and Teaching Architecture (ALTA) that 
integrates three key components: 

\begin{itemize}
    \item \textbf{Active Learning Module}: This module enables the robot to actively select informative 
    samples for learning, reducing the need for extensive labeled data. 
    \item \textbf{Teaching Module}: This component facilitates the robot's ability to teach and transfer knowledge to other agents or systems.
    \item \textbf{Adaptive Planning Module}: This module allows the robot to adapt its planning strategies based on new information and changing environments.
\end{itemize}



The overall architecture can be represented as follows:\

human demonstrate task as input to Active Learning Module output to Teaching Module 
then output to Adaptive Planning Module finally robot performs task. As human pick and place object
the robot learn from human demonstration through Active Learning Module, then Teaching Module help robot to generalize
the task to different objects, finally Adaptive Planning Module enable robot to plan the pick and place task efficiently.



The teaching module leverages techniques from \cite{dmp} to facilitate effective knowledge transfer. we utilize dreamer
to learn the world model and plan the task through adaptive planning module. The active learning module is designed to minimize human intervention by selecting the most informative samples for learning, 
as discussed in \cite{ebert2018internet}.

the integration of these components allows ALTA to effectively learn from limited data,\cite{hafner2019dream} teach other agents, and adapt to dynamic environments,

as illustrated in Figure 2. 
\begin{figure}
    \centering
    \includegraphics[width=0.8\textwidth]{../Figures/architecture.png}
    \caption{Active Learning and Teaching Architecture (ALTA) integrates Active Learning, Teaching, and Adaptive Planning modules to enable efficient robot learning and task execution.}
    \label{fig:architecture}
    
\end{figure}